% Chapter 4

\chapter{Experiments}
\label{Chapter4}

In order to assess the quality of the previously exposed ideas and approaches we
needed to test them using the framework described in section~\ref{subsec:evaluation}. 
Since our method has the parameter selection phase embedded in the learning
process (section~\ref{subsec:parameters}) the experiments structure were quite
simple, namely consisting only in the pre-calculation of the relevant kernel
matrices, one for each tuple of parameters combination, and then using them all
at once in a stratified k-fold cross-validation routine.
The $k$ parameter of the latter has been fixed at 10 since this value already
provides good statistical significance while helping containing the bias
skewing during the training phase.
Moreover, this whole routine has been run ten times to mitigate the high
variance derived during the testing phase due to the chosen value of $k$.

\section{Datasets}
\label{subsec:datasets}

The experiments were conducted on five real-world applications datasets, namely:
CAS [ref], NCI1 [ref], AIDS [ref], CPDB [ref], GDD[ref]. All these datasets
contains chemical and molecular particles encoded in graph form.
The AIDS Antiviral Screen dataset contains chemical compounds, categorized
into chemical groups; CAS and CPDB are dataset of mutagenic
compounds; NCI1 consists of chemical compounds screened for activity against 
non-small lung cancer cells; GDD is composed of X-ray crystal structures of
proteins represented as graphs.
The graph encoding of this data is such that every node is labelled and none
has a self loop that is there are no nodes with an edge going out and going back
into them.
Some statistics about these datasets are gathered in table~\ref{table:datasets}.

    \begin{table}[ht]
        \centering
        \begin{tabular}{|r|r|r|r|r|}
            \hline
            Dataset & n. of graphs & sample split & avg nodes & avg edges \\ \hline
            AIDS    & 1503         & 28.07        & 58.90     & 61.40     \\ \hline      
            CAS     & \textbf{4337} & 55.36        & 29.90     & 30.90     \\ \hline      
            CPDB    &  684         & 49.85        & 14.10     & 14.60     \\ \hline      
            GDD     & 1503         & 58.65        & 284.31    & \textbf{2862.63}   \\ \hline      
            NCI1    & 1503         & 50.04        & 29.87     & 32.30     \\ \hline      
        \end{tabular}
        \caption{Statistics about the datasets employed in the experiments: number
        of graphs, labels percentage among samples, average number of nodes, average
        number of edges. It is clear from this table that the CAS and NCI1 datasets
        are the bigger ones while the GDD holds the most complex graphs either in
        terms of topography and processing. Furthermore the AIDS dataset turns out
        to be quite unbalanced in terms of labels distribution.}
        \label{table:datasets}
    \end{table}

%----------------------------------------------------------------------------------------

\section{Description}

The kernel combinations selected for this study were:
\begin{enumerate}
    \item the $ODD_{ST}$ and $TCK_{ST}$ graph kernels,
    \item the $ODD_{ST+}$ and $TCK_{ST+}$ graph kernels,
    \item the $WL$ fast subtree and $WLC$ graph kernels,
\end{enumerate}

for each combination a separated set of experiment was conducted, while maintaining
the same structure, of which a detailed breakdown will now be given.

%\subsubsection[$ODD_{ST}$ and $ODD_{STC}$]{$\boldsymbol{ODD_{ST}}$ and $\boldsymbol{TCK_{ST}}$}
The weak kernels used for this experiment were generated from the $ODD$ kernels
in \cite{rtesselli}, namely $ODD_{ST}$, $TCK_{ST}$, $ODD_{ST+}$, and $TCK_{ST+}$.
The hyper-parameter selection process for the kernels has been embedded
into the MKL learning phase so each set of kernel matrices has been
pre-computed for each one of the possible combination of parameters that would
otherwise have been used in a grid-search fashion.
Since EasyMKL takes care of weighing the individual weak kernels, the $\lambda$
parameters of the ODD kernels has been fixed to 1, while the $h$ parameter
values were the set $\{1,\dots,10\}$.

During the cross-validation routine, the $\Lambda$ parameter of EasyMKL has been
validated from the set $\{0.0, 0.1,\dots,1.0\}$.

% brief code listing

\subsection{Alternative approach}
The routine has been implemented using a slightly modified version of EasyMKL
\cite{easymkl} in order to keep a constant level of memory occupation,
given the large amount of matrices that had to be loaded into memory at each
step of the algorithm.

% brief code listing

\section{Baselines}
To assess the quality of our results we compared them with the results we got
from testing come baseline performances against the same datasets.
Beside the performance measures available in \cite{gmkl}, we tested our framework
with only the $ODD_{ST}$ orthogonalized kernels, then with the orthogonalized
$ODD_{STC}$ kernels alone.
Furthermore we collected the same performance measure for the $ODD_{STC}$ kernel,
with and without $ST$ features, with a 10-fold cross validation implemented
using a SVM.

\subsection{MKL without buckets}

\subsection{SVM}
%----------------------------------------------------------------------------------------

\section{Results and discussion}
\label{subsec:results}

\subsection{Resources requirements}

\subsection{Models performances}

\subsection{First group}

\begin{landscape}
    \begin{table}[ht]
        \centering
        \begin{tabular}{|l|r|r|r|r|r|}
            \hline
            method&CAS&NCI1&AIDS&CPDB&GDD\\
            \hline
            MKL (ODD\_ST, TCK\_ST)* &&&0.8632 $\pm$ 0.0034&0.8632 $\pm$ 0.0033&\\
            \hline
            MKL (ODD\_ST)*&&&&&\\
            \hline
            MKL (TCK\_ST)*&&&&&\\
            \hline
            MKL (ODDS\_ST, TCK\_ST)&&&0.8468 $\pm$ 0.0041&0.8517 $\pm$ 0.0033&0.8611 $\pm$ 0.0017\\
            \hline
            MKL (ODD\_ST)&&&0.8387 $\pm$ 0.0043&0.8400 $\pm$ 0.0032&0.8012$\pm$ 0.0018\\
            \hline
            MKL (TCK\_ST)&0.8953 $\pm$ 0.0013&0.9095 $\pm$ 0.0006&0.8486 $\pm$ 0.0040&0.8525 $\pm$ 0.0029&0.8617 $\pm$ 0.0021\\
            \hline
            MKL (MKL-ODD)** &0.9049 $\pm$ 0.0008&0.9144 $\pm$ 0.0008&0.8515 $\pm$ 0.0031&0.8564 $\pm$ 0.0056&0.8498 $\pm$ 0.0026\\
            \hline
            SVM (ODD\_ST)** &0.8982 $\pm$ 0.0017&0.9069 $\pm$ 0.0010&0.8262 $\pm$ 0.0052&0.8442 $\pm$ 0.0067&0.8473 $\pm$ 0.0038\\
            \hline
            SVM (TCK\_ST +ODD\_ST)&0.9009 $\pm$ 0.0010&0.9110 $\pm$ 0.0011&0.8322 $\pm$ 0.0065&0.8497 $\pm$ 0.0072&0.8626 $\pm$ 0.0018\\
            \hline
            SVM (TCK\_ST)&0.9006 $\pm$ 0.0013&0.9149 $\pm$ 0.0010&0.8225 $\pm$ 0.0066&0.8421 $\pm$ 0.0080&0.8674 $\pm$ 0.0026\\
            \hline
        \end{tabular}
        \label{table:results_st}
        \caption{ROAUC measure relative to the first experiment. }

        \centering
        \begin{tabular}{|l|r|r|r|r|r|}
            \hline
            method&CAS&NCI1&AIDS&CPDB&GDD\\
            \hline
            MKL ($ODD_{ST+}, TCK_{ST+}$)*&&&0.8632 $\pm$ 0.0034&0.8632 $\pm$ 0.0033&\\
            \hline
            ODD\_ST con MKL*&&&&&\\
            \hline
            TCK\_ST con MKL*&&&&&\\
            \hline
            TCK\_ST (+ODD\_ST) con SVM&0.9009 $\pm$ 0.0010&0.9110 $\pm$ 0.0011&0.8322 $\pm$ 0.0065&0.8497 $\pm$ 0.0072&0.8626 $\pm$ 0.0018\\
            \hline
            TCK\_ST con SVM&0.9006 $\pm$ 0.0013&0.9149 $\pm$ 0.0010&0.8225 $\pm$ 0.0066&0.8421 $\pm$ 0.0080&0.8674 $\pm$ 0.0026\\
            \hline
            ODDS\_ST, TCK\_ST con MKL&&&0.8468 $\pm$ 0.0041&0.8517 $\pm$ 0.0033&0.8611 $\pm$ 0.0017\\
            \hline
            ODD\_ST con MKL&&&0.8387 $\pm$ 0.0043&0.8400 $\pm$ 0.0032&0.8012$\pm$ 0.0018\\
            \hline
            TCK\_ST con MKL&0.8953 $\pm$ 0.0013&0.9095 $\pm$ 0.0006&0.8486 $\pm$ 0.0040&0.8525 $\pm$ 0.0029&0.8617 $\pm$ 0.0021\\
            \hline
        \end{tabular}
        \label{table:results_stp}
        \caption{ROAUC measure relative to the second experiment. }
    \end{table}

    \begin{table}[ht]
        \centering
        \begin{tabular}{|l|r|r|r|r|r|}
            \hline
            method&CAS&NCI1&AIDS&CPDB&GDD\\
            \hline
            ODD\_ST, TCK\_ST con MKL*&&&0.8632 $\pm$ 0.0034&0.8632 $\pm$ 0.0033&\\
            \hline
            ODD\_ST con MKL*&&&&&\\
            \hline
            TCK\_ST con MKL*&&&&&\\
            \hline
            TCK\_ST (+ODD\_ST) con SVM&0.9009 $\pm$ 0.0010&0.9110 $\pm$ 0.0011&0.8322 $\pm$ 0.0065&0.8497 $\pm$ 0.0072&0.8626 $\pm$ 0.0018\\
            \hline
            TCK\_ST con SVM&0.9006 $\pm$ 0.0013&0.9149 $\pm$ 0.0010&0.8225 $\pm$ 0.0066&0.8421 $\pm$ 0.0080&0.8674 $\pm$ 0.0026\\
            \hline
            ODDS\_ST, TCK\_ST con MKL&&&0.8468 $\pm$ 0.0041&0.8517 $\pm$ 0.0033&0.8611 $\pm$ 0.0017\\
            \hline
            ODD\_ST con MKL&&&0.8387 $\pm$ 0.0043&0.8400 $\pm$ 0.0032&0.8012$\pm$ 0.0018\\
            \hline
            TCK\_ST con MKL&0.8953 $\pm$ 0.0013&0.9095 $\pm$ 0.0006&0.8486 $\pm$ 0.0040&0.8525 $\pm$ 0.0029&0.8617 $\pm$ 0.0021\\
            \hline
        \end{tabular}
        \label{table:results_wl}
        \caption{ROAUC measure relative to the third experiment. }
    \end{table}
\end{landscape}

\subsection{Second group}
\subsection{Third group}

\section{Kernels Contribution Breakdown}
% eventuale plot dei pesi delle matrici per gli esperimenti principali

