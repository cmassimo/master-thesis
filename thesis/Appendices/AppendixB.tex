\chapter{Language and Libraries}
\label{AppendixB}

\section{Language}
\label{sec:language}

The main language used throughout the implementation phase of this work is 
Python.
The language has been chosen for the good tradeoff between ease of use and
performances and because most of the available code in terms of kernel functions 
and learning methods is written in Python.
In the recent year the language has established itself as a valuable tool for
scientific computing, especially for machine learning, thanks to a wealth
of libraries specifically devoted to the task.
A brief overview of the libraries employed to implement the methodology
and the experiments is given in the following section.

\section{Libraries}
\label{sec:libraries}
The main third-party libraries used in this work are:
\begin{itemize}
    \item scikit-learn (\url{http://scikit-learn.org/stable/index.html}) for the
        SVM implementation and matrices I/O;
    \item EasyMKL (\url{https://github.com/jmikko/EasyMKL/}) for the chosen $MKL$
        implementation;
    \item cvxopt (\url{http://cvxopt.org/}) for dataset processing since they are
        a pre-requisite of EasyMKL;
    \item matplotlib (\url{http://matplotlib.org}) \cite{Hunter:2007} for all the plots;
    \item numpy (\url{http://www.numpy.org/}) for everything else.
\end{itemize}

\subsection{Original code, and data repositories}
\label{subsec:repos}
Some of the original code produced during this work is available as a branch of the 
very useful scikit-learn-graph library (\url{https://github.com/nickgentoo/scikit-learn-graph})
put together by Nicol\`o Navarin which contained most of the methods and kernel
functions on which this work has been based.
The code can be viewed and downloaded here: \url{https://github.com/nickgentoo/scikit-learn-graph/tree/thesis-cmassimo}.
This repository contains mainly integrated and revised versions of the kernel functions from \cite{rtesselli},
plus all the novel implementations proposed in this study.
The repository also contains the kernel functions proposed in \cite{SanMartino2014}
as well as the integration of the weighing scheme from \cite{DaSanMartino2016} for all the $ODD$ kernels,
both of which did not make it past the initial exploratory phase.

The code for the experiments and the actual scripts with the methodology implementations
can be browsed and downloaded here: \url{https://github.com/cmassimo/master-thesis}.

The datasets used in this work are publicly available and can be downloaded from:
\begin{itemize}
    \item \url{https://sites.google.com/site/nicknavarin/software/datasets}
    \item \url{http://www.math.unipd.it/~nnavarin/datasets/}
\end{itemize}
