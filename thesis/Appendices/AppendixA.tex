\chapter{A further baseline}
\label{AppendixA}


\section{Hierarchical MKL}
\label{subsec:hierarchy}
% describe the approach in detail (diagram and explaination)


\begin{figure}[ht]
    \centering
    \includegraphics[scale=0.5]{Figures/nested}
    \caption{The method illustrated. First the dataset, composed of a list of
    kernel matrices get split into a training set and a test set (1).
    Then a $k$-fold cross-validation is performed on the training set ($k$ was set to
    10 in our case) (2) and the best resulting model is selected according to some
    performance measure (3). The selected model gets re-trained on the whole initial
    training set (4) and finally gets tested against the test set that was left out
    during the entire process (5). At the end of the outer loop of the nested
    cross-validation, the best model is again selected according to the chosen
    metric (6).}
    \label{fig:hmethod}
\end{figure}

% vim: spell spelllang=en_gb
