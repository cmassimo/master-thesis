\chapter{A further baseline}
\label{AppendixA}


\section{Hierarchical MKL}
\label{subsec:hierarchy}
% describe the approach in detail (diagram and explaination)

Here we will give a brief description of another baseline method that was implemented
but not tested due to time constraints.
This baseline aims to determine if dividing the combination process in two steps, i.e.
first combine only the kernel generated from the same function and then combine
the weighed sum, could boost the contribution of the single function w.r.t.
the methodology proposed in this work.
This baseline method is a good candidate to be tested in future work related
to this thesis.

\begin{figure}[h]
    \centering
    \includegraphics[scale=0.45]{Figures/hierarchy}
    \caption{\footnotesize The baseline illustrated. The depicted scenario considers
        a single learning phase, where $M > 1$ kernel functions are combined, each giving
        origin to $N$ kernel matrices, though this number can certainly vary
        from function to function depending on the size of the parameters grid, grouped in $M$ sets (a).
        Each set is trained independently using EasyMKL (Section \ref{subsec:easymkl})
        and then used to compute a weighed sum using the weights returned by EasyMKL (b).
        The sum kernels thus obtained ($S_1,\dots,S_M$) are used in combination to train another model (c).
    }
    \label{fig:hmethod}
\end{figure}

% vim: spell spelllang=en_gb
